
\newlength{\smallcol}
\setlength{\smallcol}{0.333333333333\textwidth}

\newlength{\bigcol}
\setlength{\bigcol}{\textwidth}
\addtolength{\bigcol}{- \smallcol}


\begin{frame}[plain]
  \mode<article>{\maketitle}
  \mode<presentation>{\titlepage}
\end{frame}

\section{what is hgg}

\subsection{History}
	\begin{frame}{CCCamp 2011}
		\begin{itemize}
			\item Nick Farr, Lars Weiler, Jens Ohlig propose a "Hacker Space Program"
			\item Three hackers from shackspace immediately brainfart
			\item "This is awesome!" -> "Let's do it!"
		\end{itemize}
	\end{frame}
	\begin{frame}{Joining up w/ Constellation}
		\begin{itemize}
			\item Andreas Hornig of AerospaceResearch.net ends up giving a talk on Constellation at shackspace
			\item Both sides immediately notice the similarity in his DGSN and our HGG idea
			\item We join forces
		\end{itemize}
	\end{frame}
	\begin{frame}{"Call to arms" talk at 28c3}
		\begin{itemize}
			\item After the initial research and proof of concepts we thought it would be nice to have 3 to 5 more folks helping us
			\item So we handed in a talk for 28c3
			\item Press feedback was never the same...
		\end{itemize}
	\end{frame}
	\begin{frame}{Press feedback or "What hgg definitely isn't"}
	\end{frame}

\subsection{hgg in a nutshell}
	\begin{frame}{build a modular system}
  	\begin{columns}
    	\begin{column}{\smallcol}
 				\begin{center}\includegraphics<1->[width=\textwidth]{modular}\end{center}
			\end{column}
    	\begin{column}{\bigcol}
				\begin{itemize}
					\item Easier to develop
					\item Easier to extend
					\item Easier to improve
				\end{itemize}
			\end{column}
		\end{columns}
	\end{frame}
	\begin{frame}{make it as accurate as possible}
  	\begin{columns}
    	\begin{column}{\smallcol}
			\end{column}
    	\begin{column}{\bigcol}
				\begin{itemize}
					\item 1 s resolution is "boring"
					\item Let's aim for 100 ns
					\item Allow scaling up to "ridiculous"
				\end{itemize}
			\end{column}
		\end{columns}
	\end{frame}
	\begin{frame}{measure stuff}
  	\begin{columns}
    	\begin{column}{\smallcol}
			\end{column}
    	\begin{column}{\bigcol}
				\begin{itemize}
					\item Airplanes
					\item Satellites
					\item Or even just the temperate
				\end{itemize}
			\end{column}
		\end{columns}
	\end{frame}
	\begin{frame}{make it a distributed system}
  	\begin{columns}
    	\begin{column}{\smallcol}
				\begin{center}\includegraphics<1->[width=\textwidth]{distributed}\end{center}
			\end{column}
    	\begin{column}{\bigcol}
				\begin{itemize}
					\item Many simple measurement stations
					\item networked together
					\item providing geo-coded data
				\end{itemize}
			\end{column}
		\end{columns}
	\end{frame}
	\begin{frame}{make it easy to use}
  	\begin{columns}
    	\begin{column}{\smallcol}
 				\begin{center}\includegraphics<1->[width=\textwidth]{easytouse}\end{center}
			\end{column}
    	\begin{column}{\bigcol}
				\begin{itemize}
					\item Ideal: build your own
					\item Realistic: assemble a kit
					\item Lazy: buy it, plug it, forget about it
				\end{itemize}
			\end{column}
		\end{columns}
	\end{frame}

\subsection{Who's behind it?}
	\begin{frame}{Who's behind it?}
		\begin{itemize}
			\item Just a bunch of folks, really
			\item No company or governments
			\item By hackers, for everyone
		\end{itemize}
	\end{frame}


\section{What we're actually doing}

\subsection{The core idea}
	\begin{frame}{Consolidating existing and new information}
	\end{frame}
	\begin{frame}{Learning the basics}
		\begin{itemize}
			\item PCB design
			\item FPGA programming in VHDL
			\item Microcontroller programming in C
			\item Antenna design
		\end{itemize}
	\end{frame}
	\begin{frame}{Open source everything}
	\end{frame}
	\begin{frame}{What is it actually good for?}
		\begin{itemize}
			\item Public access to all measurement results (don't get cheated)
			\item Access to infrastructure to deploy own (measurement) equipment
		\end{itemize}
	\end{frame}
	\begin{frame}{Are there already applications?}
		Constellation
	\end{frame}

\subsection{Status quo}
  \begin{frame}{Specification of physical interface between modules}
		\begin{itemize}
			\item Modules are connected via a backplane
			\item PCIe 4x plug w/ custom pinout
			\item 2x RS485 lanes for inter-module communication
			\item SPI-ish time broadcast bus
			\item Differential clock signal for high-res clock signal
		\end{itemize}
	\end{frame}
	\begin{frame}{friendship0 backplane}
		\begin{itemize}
			\item 4 modules slots, one dedicated to bus master module
			\item ICs for interrupt handling
		\end{itemize}
	\end{frame}
	\begin{frame}{braeburn0 power supply module}
		\begin{itemize}
			\item Single external power source
			\item All voltages generated on-board, stabilized
			\item In-system voltage level monitoring
		\end{itemize}
	\end{frame}
	\begin{frame}{braeburn1 power supply module}
		\begin{itemize}
			\item Uses PC power supply
			\item Minor stabilization / buffering
			\item In-system voltage level monitoring
		\end{itemize}
	\end{frame}
	\begin{frame}{flutter0 high precision distributed time source module}
		\begin{itemize}
			\item Spartan 3A FPGA for high-res timing (<100 ns)
			\item ATmega 168 for lo-res timing (1 s to 1/10th s)
			\item Low cost GPS module w/ external antenna support
		\end{itemize}
	\end{frame}
	\begin{frame}{dash0 proof of concept}
		\begin{itemize}
			\item ADS-B receiver based around miniADSB module
			\item Easily track commercial aircraft
			\item Perfect for verifying pseudo ranging algorithms
		\end{itemize}
	\end{frame}

\section{On the horizon}	

\subsection{Roadmap}
	\begin{frame}{celestia0 bus master module}
		\begin{itemize}
			\item Manages interrupt requests by modules
			\item Arbitrates resources
			\item Enumeration of available modules
		\end{itemize}
	\end{frame}
	\begin{frame}{dash0 ADSB receiver module}
		\begin{itemize}
			\item Built around the proof of concept
			\item Most likely CPLD-based decoding of Manchester-encoded signal
			\item Contributions by Pawel
		\end{itemize}
	\end{frame}
	\begin{frame}{magic0 bus protocol}
		\begin{itemize}
			\item What modules will speak between each other and the bus master
			\item Data exchange and enumeration
		\end{itemize}
	\end{frame}
	\begin{frame}{Testing timing accuracy}
		\begin{itemize}
			\item First level test: 2x ground stations w/ flutter module
			\item Second level test: ~5 ground stations w/ flutter module
		\end{itemize}
	\end{frame}
	\begin{frame}{Calibration}
		\begin{itemize}
			\item High accuracy measurement requires diligent calibration
			\item Receiver, decoder, communication lags
			\item Phase error
			\item ...
		\end{itemize}
	\end{frame}
	\begin{frame}{Deploying 3 to 5 systems}
		\begin{itemize}
			\item Test pseudo ranging and timing
			\item This will be the make or break decision
		\end{itemize}
	\end{frame}
	\begin{frame}{Quality tests and review}
		\begin{itemize}
			\item Review everything
			\item Make improvements where necessary
			\item Manufacture pre-series
			\item Hand ground stations out to other hackerspaces and interested parties
		\end{itemize}
	\end{frame}
	\begin{frame}{More modules}
		\begin{itemize}
			\item Arduino module
			\begin{itemize}
				\item Probably the easiest way to prototype
				\item Make it available to an already large community
			\end{itemize}
			\item Environment sensors
			\begin{itemize}
				\item Measure ALL the things
				\item Temperature, humidity, barometric pressure, seismic waves, radiation, tectonic drift, time, wind, ...
			\end{itemize}
		\end{itemize}
	\end{frame}
	\begin{frame}{satellites}
		\begin{itemize}
			\item Not impossible, though not really /our/ goal
		\end{itemize}
	\end{frame}

\subsection{how to help}
	\begin{frame}{Why we have not asked for donations, yet}
		\begin{itemize}
			\item Offers from heartwarming to ridiculous
			\item Still doing research and feasibility studies
			\item No guarantee that it'll ever work (chances are good, though)
			\item No money asked, no one disgruntled if it fails.
		\end{itemize}
	\end{frame}
	\begin{frame}{When we might ask for money}
		\begin{itemize}
			\item After prototype works good enough
			\item Before rolling out on a bigger scale (think 10+)
		\end{itemize}
	\end{frame}
	\begin{frame}{hgg.aero}
		\begin{itemize}
			\item Wiki for everyone to join in
			\item List of open tasks.  Pick one or add one!
		\end{itemize}
	\end{frame}
	\begin{frame}{github.com/shackspace/hgg}
		\begin{itemize}
			\item All source code, schematics and layouts available
			\item Issue tracking.  Find a problem, raise an issue!
		\end{itemize}
	\end{frame}
	\begin{frame}{lists.shackspace.de/listinfo/constellation}
		\begin{itemize}
			\item Public mailing list, feel free to join.
			\item fairly low traffic at the moment, this might change in the foreseeable future.
		\end{itemize}
	\end{frame}



% 
% \section{Warum dieses Projekt?}
% \begin{frame}{Warum dieses Projekt?}
%   \begin{itemize}
%     \item Sicherheitslücke im bisherigen system
%     \item nur eine Lösung: \alert<.>{das muss gehackt werden} 
%   \end{itemize}
% \end{frame}
% 
% \section*{Anforderungen an das Projekt}
% \begin{frame}{Anforderungen an das Projekt}
%   \begin{itemize}
%     \item mehrere Schlüssel müssen getrackt werden
%     \item Schlüssel müssen individuell getrackt werden können
%     \item fehlender Schlüssel muss sich durch ein akustisches Signal bemerkbar machen
%     \begin{itemize}
%       \item mit einem Taster muss für ein paar Minuten wider Ruhe einkehren
%     \end{itemize}
%     \item Schlüssel müssen dem System einfach hinzugefügt und entfernt werden können
%     \item Schlüssel mit besonderer Funktion müssen gesondert behandelt werden können
%     \begin{itemize}
%       \item z.B. \shack{}-Schlüssel
%     \end{itemize}
%   \end{itemize}
% \end{frame}
% 
% \section{Umsetztung}
% \begin{frame}[<.->]{Umsetzung}
%   \begin{columns}
%     \begin{column}{\bigcol}
%       \begin{itemize}
%         \item<+-> Patchpanel (8 Ports)
%         \item<+-> Netzwerkkabel als Schlüsselbund
%         \item<+-> Logik auf Basis des shackuino
%       \end{itemize}
%     \end{column}
%     \begin{column}{\smallcol}
%       \includegraphics<1->[width=\smallcol]{patchpanel}
%     \end{column}
%   \end{columns} 
% \end{frame}
% 
% 
% \section{Stand}
% \begin{frame}[<.->]{Stand}
%   \begin{columns}
%     \begin{column}{\bigcol}
%       \begin{itemize}
%         \item<+-> Prototyp auf Breadboard
%         \item<+-> shackuino
%         \begin{itemize}
%           \item<+-> Code mit absoluten Grundanforderungen
%         \end{itemize}
%         \item<+-> Netzwerkdose aus dem alten \shack
%         \item<+-> zwei einseitig gekrimpte CAT5 Verlegekabel
%       \end{itemize}
%     \end{column}
%     \begin{column}{\smallcol}
%       \includegraphics<1-3>[width=\smallcol]{board}
%       \includegraphics<4-5>[width=\smallcol]{netzwerkdose}
%     \end{column}
%   \end{columns}
% \end{frame}
% 
% 
% 
% \section{Probleme}
% \begin{frame}{Probleme}
%   \begin{itemize}
%     \item Wackelkontakte im Breadboard
%     \item Verlegekabel lassen sich nicht krimpen
%     \item in Netzwerkdosen kann man eigentlich nur ein kabel auflegen
%     \item Millis() gibt ein unsigned long zurück. Sollte nicht als int benutzt werden.
%   \end{itemize}
% \end{frame}
% 
% \section{ToDo}
% \begin{frame}{ToDo}
%   \begin{itemize}
%     \item Code vervollständigen
%     \item auf Bauteile warten
%     \begin{itemize}
%       \item Platz in Patchpanel validieren
%     \end{itemize}
%     \item Schaltplan
%     \item Eagle-Design
%     \item Platine ätzen
%     \item Platine in Patchpanel einnbauen
%     \item im \shack{} aufhängen
%   \end{itemize}
% \end{frame}
% 
% \section{zukünftige Probleme}
% \begin{frame}{zukünftige Probleme}
%   \begin{itemize}
%     \item \alert<.>{Ich hab keine Ahnung von Mikrocontroller-Programmierung}
%     \item \alert<.>{ich hab keine Ahnung von Elektrokram}
%     \begin{itemize}
%       \item bzw. alles wider vergessen seit der Schule
%     \end{itemize}
%     \item \alert<.>{ich hab keine Ahnung von Eagle}
%     \item \alert<.>{ich hab keine Ahnung vom Ätzen}
%     \item \alert<.>{meine Fantasie hubelt die ganze zeit}
%   \end{itemize}
% \end{frame}
% 
% %\begin{frame}{ENDE}
% %  \begin{center}
% %    \includegraphics[width=.5\textwidth]{logo_shack_brightbg} \\
% %    Vielen dank für eure Aufmerksamkeit
% %  \end{center}
% %\end{frame}
% 
% \date{ }
% 
% 
% \titlegraphic{Vielen Dank für eure Aufmerksamkeit}
% 
% \begin{frame}[plain]
%   \mode<article>{\maketitle}
%   \mode<presentation>{\titlepage}
% \end{frame}
