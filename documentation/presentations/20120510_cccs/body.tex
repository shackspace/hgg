
\newlength{\smallcol}
\setlength{\smallcol}{0.333333333333\textwidth}

\newlength{\bigcol}
\setlength{\bigcol}{\textwidth}
\addtolength{\bigcol}{- \smallcol}


\begin{frame}[plain]
  \mode<article>{\maketitle}
  \mode<presentation>{\titlepage}
\end{frame}

\section{What is hgg}

\subsection{History}
	\begin{frame}{CCCamp 2011}
		\begin{itemize}
			\item Nick Farr, Lars Weiler, Jens Ohlig propose a "Hacker Space Program"
			\item Three hackers from shackspace immediately brainfart
			\item "This is awesome!"
			\item "Let's do it!"
		\end{itemize}
	\end{frame}
	\begin{frame}{Joining up w/ Constellation}
		\begin{itemize}
			\item Andreas Hornig of AerospaceResearch.net ends up giving a talk on Constellation at shackspace
			\item Both sides immediately notice the similarity in his DGSN and our HGG idea
			\item We join forces
		\end{itemize}
	\end{frame}
	\begin{frame}{"Call to arms" talk at 28c3}
		\begin{itemize}
			\item After the initial research and proof of concepts we thought it would be nice to have 3 to 5 more folks helping us
			\item So we handed in a talk for 28c3
			\item Press feedback was never the same...
		\end{itemize}
	\end{frame}
	\begin{frame}{Press feedback or "What hgg definitely isn't"}
		\begin{itemize}
			\item "Hacker aus Stuttgart - Mit dem Lötkolben ins Weltall"\\ \footnotesize{\emph{-- Stuttgarter Zeitung}}
			\item "Hacking im Weltraum - Hacker arbeiten an eigenem Satellitennetzwerk"\\ \footnotesize{\emph{-- Golem}}
			\item "Hackers send internet into space"\\ \footnotesize{\emph{-- UK Metro}}
			\item "Hackers plan space satellites to combat censorship"\\ \footnotesize{\emph{-- bbc}}
			\item "Hacking confab conjures visions of space-borne 'SOPA Wars'"\\ \footnotesize{\emph{-- cnet}}
		\end{itemize}
	\end{frame}

\subsection{hgg in a nutshell}
	\begin{frame}[<.->]{Build a modular system}
  	\begin{columns}
    	\begin{column}{\smallcol}
 				\begin{center}\includegraphics<1->[width=\textwidth]{modular}\end{center}
			\end{column}
    	\begin{column}{\bigcol}
				\begin{itemize}
					\item<+-> Easier to develop
					\item<+-> Easier to extend
					\item<+-> Easier to improve
				\end{itemize}
			\end{column}
		\end{columns}
	\end{frame}
	\begin{frame}[<.->]{Make it as accurate as possible}
  	\begin{columns}
    	\begin{column}{\smallcol}
			\end{column}
    	\begin{column}{\bigcol}
				\begin{itemize}
					\item<+-> One second resolution is "boring"
					\item<+-> Let's aim for 100 ns
					\item<+-> Allow scaling up to "ridiculous"
				\end{itemize}
			\end{column}
		\end{columns}
	\end{frame}
	\begin{frame}[<.->]{Measure stuff}
  	\begin{columns}
    	\begin{column}{\smallcol}
			\end{column}
    	\begin{column}{\bigcol}
				\begin{itemize}
					\item<+-> Airplanes
					\item<+-> Satellites
					\item<+-> Or even just the temperate
				\end{itemize}
			\end{column}
		\end{columns}
	\end{frame}
	\begin{frame}[<.->]{Make it a distributed system}
  	\begin{columns}
    	\begin{column}{\smallcol}
				\begin{center}\includegraphics<1->[width=\textwidth]{distributed}\end{center}
			\end{column}
    	\begin{column}{\bigcol}
				\begin{itemize}
					\item<+-> Many simple measurement stations
					\item<+-> networked together
					\item<+-> providing geo-coded data
				\end{itemize}
			\end{column}
		\end{columns}
	\end{frame}
	\begin{frame}[<.->]{Make it easy to use}
  	\begin{columns}
    	\begin{column}{\smallcol}
 				\begin{center}\includegraphics<1->[width=\textwidth]{easytouse}\end{center}
			\end{column}
    	\begin{column}{\bigcol}
				\begin{itemize}
					\item<+-> Ideal: build your own
					\item<+-> Realistic: assemble a kit
					\item<+-> Lazy: buy it, plug it, forget about it
				\end{itemize}
			\end{column}
		\end{columns}
	\end{frame}

\subsection{Who's behind it?}
	\begin{frame}{Who's behind it?}
		\begin{itemize}
			\item Just a bunch of folks, really
				\begin{itemize}
					\item reloc0 \& hadez \& saeugetier working on hgg
					\item horn working on Constellation
					\item Pawe\l, Isaac, and a few others working on various projects
				\end{itemize}
			\item No company or governments
			\item By hackers, for everyone
		\end{itemize}
	\end{frame}


\section{What we're actually doing}

\subsection{The core idea}
	\begin{frame}{Consolidating existing and new information}
		\begin{itemize}
			\item There is already \emph{a lot} of information available
			\begin{itemize}
				\item HAM radio community
				\item Amateur satellite community
				\item Hackers \& makers
			\end{itemize}
			\item We're collecting information relevant to the ask
			\item Try to make it easier to understand where certain details aren't documented well
			\item Document our findings, results and failures for others to learn from
		\end{itemize}
	\end{frame}
	\begin{frame}{Learning the basics}
		\begin{itemize}
			\item PCB design
			\item FPGA programming in VHDL
			\item Microcontroller programming in C
			\item Antenna design
		\end{itemize}
	\end{frame}
	\begin{frame}{Open source everything}
		\begin{itemize}
			\item Code available at github.com/shackspace/hgg
			\item Documentation and planning at hgg.aero/
		\end{itemize}
	\end{frame}
	\begin{frame}{What is it actually good for?}
		\begin{itemize}
			\item Public access to all measurement results (don't get cheated)
			\item Access to infrastructure to deploy own (measurement) equipment
		\end{itemize}
	\end{frame}
	\begin{frame}{What about applications?}
		\begin{itemize}
			\item Constellation
			\begin{itemize}
				\item Track amateur satellites
				\item Using pseudo-ranging w/ multiple receiver stations
			\end{itemize}
			\item Once ground stations start gathering and publishing data, the possibilities are endless
			\begin{itemize}
				\item Live-track background radiation levels
				\item Spot minute changes in the environment over time
				\item Accurate, geo-referenced time
				\item Basis for assisted GPS solutions
				\item and many, many more
			\end{itemize}
		\end{itemize}
	\end{frame}

\subsection{Status quo}
  \begin{frame}{Specification of physical interface between modules}
		\begin{itemize}
			\item Modules are connected via a backplane
			\item PCIe 4x plug w/ custom pinout
			\item 2x RS485 lanes for inter-module communication
			\item SPI-ish time broadcast bus
			\item Differential clock signal for high-res timing signal
			\item Each module sports storage for calibration data
		\end{itemize}
	\end{frame}
	\begin{frame}{friendship0 backplane}
		\begin{itemize}
			\item Four modules slots, one dedicated to bus master module
			\item ICs for interrupt handling
			\item Can be easily scaled up, next step eight or nine slots
		\end{itemize}
	\end{frame}
	\begin{frame}{braeburn0 power supply module}
		\begin{itemize}
			\item Single external power source
			\item All voltages generated on-board, stabilized
			\item In-system voltage level monitoring
		\end{itemize}
	\end{frame}
	\begin{frame}{braeburn1 power supply module}
		\begin{itemize}
			\item Uses PC power supply
			\item Minor stabilization / buffering
			\item In-system voltage level monitoring
		\end{itemize}
	\end{frame}
	\begin{frame}{flutter0 high precision distributed time source module}
		\begin{itemize}
			\item Ssubsubsectionan 3A FPGA for high-res timing (<100 ns)
			\item ATmega 168 for lo-res timing (1 s to 1/10th s)
			\item Low cost GPS module w/ external antenna support
		\end{itemize}
	\end{frame}
	\begin{frame}{dash0 proof of concept}
		\begin{itemize}
			\item ADS-B receiver based around miniADSB module
			\item Easily track commercial aircrafts
			\item Perfect for verifying pseudo ranging algorithms
		\end{itemize}
	\end{frame}

\section{On the horizon}	

\subsection{Roadmap}
	\begin{frame}{celestia0 bus master module}
		\begin{itemize}
			\item Manages interrupt requests by modules
			\item Arbitrates resources
			\item Enumeration of available modules
		\end{itemize}
	\end{frame}
	\begin{frame}{dash0 ADSB receiver module}
		\begin{itemize}
			\item Built around the proof of concept
			\item Most likely CPLD-based decoding of Manchester-encoded signal
			\item Contributions by Pawel
			\item Perfect to test pseudo-ranging because ADSB signal contains GPS location data already (ground truth)
			\item Your own fligt tracking radar at home?  Hell, yeah!
		\end{itemize}
	\end{frame}
	\begin{frame}{magic0 bus protocol}
		\begin{itemize}
			\item Protocol spoken between modules and master
			\item Handles data exchange and enumeration
		\end{itemize}
	\end{frame}
	\begin{frame}{Testing timing accuracy}
		\begin{itemize}
			\item First level test: 2x ground stations w/ flutter module
			\item Second level test: ~5 ground stations w/ flutter module
		\end{itemize}
	\end{frame}
	\begin{frame}{Calibration}
		\begin{itemize}
			\item High accuracy measurement requires diligent calibration
			\item Receiver, decoder, communication lags
			\item Phase error
			\item ...
		\end{itemize}
	\end{frame}
	\begin{frame}{Deploying 5+ systems}
		\begin{itemize}
			\item Test pseudo ranging and timing
			\item This will decide whether tracking would already work with our timing resolution
			\item If not, timing resolution could be scaled up by factor 10 easily
		\end{itemize}
	\end{frame}
	\begin{frame}{Quality tests and review}
		\begin{itemize}
			\item Review everything
			\item Make improvements where necessary
			\item Manufacture pre-series
			\item Hand ground stations out to other hackerspaces and interested subsubsectionies
		\end{itemize}
	\end{frame}
	\begin{frame}{More modules}
		\begin{itemize}
			\item Arduino module
			\begin{itemize}
				\item Probably the easiest way to prototype
				\item Make it available to an already large community
			\end{itemize}
			\item Environment sensors
			\begin{itemize}
				\item Measure ALL the things
				\item Temperature, humidity, barometric pressure, seismic waves, radiation, tectonic drift, time, wind, ...
			\end{itemize}
		\end{itemize}
	\end{frame}
	\begin{frame}{Satellites!}
		\begin{itemize}
			\item Not impossible, though not really \emph{our} goal
		\end{itemize}
	\end{frame}

\subsection{How to help}
	\begin{frame}{Why we have not asked for donations, yet}
		\begin{itemize}
			\item Offers from heartwarming to ridiculous
			\item Still doing research and feasibility studies
			\item No guarantee that it'll ever work (chances are good, though)
			\item No money asked, no one disgruntled if it fails.
		\end{itemize}
	\end{frame}
	\begin{frame}{When we might ask for money}
		\begin{itemize}
			\item After prototype works good enough
			\item Before rolling out on a bigger scale (think 10+)
		\end{itemize}
	\end{frame}
	\begin{frame}{Keep in touch}
		\begin{itemize}
			\item Wiki
			\begin{itemize}
				\item Edit away at http://hgg.aero/
				\item There's a list of open tasks.  Pick one or add one!
			\end{itemize}

			\item GitHub
			\begin{itemize}
				\item All source code, schematics and layouts available at github.com
				\item Issue tracking.  Find a problem, raise an issue!
			\end{itemize}

			\item Public mailing list
			\begin{itemize}
				\item lists.shackspace.de/listinfo/constellation
				\item Fairly low traffic at the moment, this might change in the foreseeable future.
			\end{itemize}

			\item twitter
			\begin{itemize}
				\item @hxglobalgrid
			\end{itemize}
		\end{itemize}
	\end{frame}


% 
% %\begin{frame}{ENDE}
% %  \begin{center}
% %    \includegraphics[width=.5\textwidth]{logo_shack_brightbg} \\
% %    Vielen dank für eure Aufmerksamkeit
% %  \end{center}
% %\end{frame}
% 
% \date{ }
% 
% 
% \titlegraphic{Vielen Dank für eure Aufmerksamkeit}
% 
% \begin{frame}[plain]
%   \mode<article>{\maketitle}
%   \mode<presentation>{\titlepage}
% \end{frame}
